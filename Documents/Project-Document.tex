\documentclass{article}
\usepackage[utf8]{inputenc}

\title{CentralAuth: Microservicio Centralizado de Autenticación y Gestión+ de Usuarios}
\author{Oscar Andres Vargas Zazzali}
\date{25 de junio de 2024}

\begin{document}

\maketitle

\begin{center}
    \textbf{Versión del Documento: 1.0}
\end{center}
\newpage
\tableofcontents
\newpage

\section{Resumen Ejecutivo}

El proyecto \textbf{CentralAuth} consiste en desarrollar un microservicio centralizado para la autenticación y gestión de usuarios. Este microservicio permitirá la centralización de los datos de usuarios, mejorando la seguridad y eficiencia en la gestión de identidades. Además, ofrecerá funcionalidades como el registro de usuarios, inicio de sesión, recuperación de contraseñas y gestión de roles y permisos. El objetivo es proporcionar una solución escalable y reutilizable que pueda integrarse fácilmente en múltiples aplicaciones y servicios.

\subsection{Objetivos del Proyecto}

\begin{itemize}
    \item \textbf{Objetivo General:} Desarrollar un microservicio centralizado para la autenticación y gestión de usuarios, que ofrezca seguridad, escalabilidad y eficiencia en la administración de identidades.
    \item \textbf{Objetivos Específicos:}
          \begin{itemize}
              \item Centralización de Datos: Unificar la gestión de usuarios y autenticación en un único sistema centralizado para facilitar el mantenimiento y la administración.
              \item Seguridad: Implementar medidas de seguridad avanzadas, incluyendo cifrado de contraseñas, autenticación de dos factores (2FA) y protección contra ataques comunes como el phishing y el brute force.
              \item Escalabilidad: Diseñar el sistema de manera que pueda escalar horizontalmente para manejar un gran número de usuarios y solicitudes simultáneas.
              \item Interoperabilidad: Garantizar que el microservicio pueda integrarse fácilmente con diversas aplicaciones y plataformas mediante API RESTful.
              \item Gestión de Roles y Permisos: Proveer funcionalidades avanzadas para la gestión de roles y permisos, permitiendo una administración granular de las autorizaciones de usuario.
              \item Experiencia de Usuario: Asegurar una experiencia de usuario intuitiva y fluida tanto para los administradores como para los usuarios finales.
          \end{itemize}
\end{itemize}

\subsection{Importancia del Proyecto}

El proyecto \textbf{CentralAuth} es crucial por varias razones:

\begin{itemize}
    \item \textbf{Seguridad Mejorada:} Al centralizar la autenticación y gestión de usuarios, se pueden implementar medidas de seguridad robustas de manera consistente en todas las aplicaciones y servicios que utilicen el microservicio. Esto reduce el riesgo de brechas de seguridad y garantiza una protección uniforme de los datos de los usuarios.
    \item \textbf{Eficiencia Operativa:} Centralizar la administración de usuarios permite un manejo más eficiente de las identidades, roles y permisos. Los administradores pueden gestionar todos los usuarios desde un único punto, reduciendo la redundancia y el esfuerzo manual asociado con la gestión de múltiples sistemas de autenticación.
    \item \textbf{Escalabilidad y Flexibilidad:} Al ser un microservicio independiente, \textbf{CentralAuth} puede escalarse horizontalmente para manejar un aumento en la carga de trabajo, lo que es esencial para aplicaciones que experimentan un crecimiento rápido en el número de usuarios.
    \item \textbf{Reutilización y Consistencia:} Este microservicio puede ser reutilizado por diferentes aplicaciones dentro de una organización, asegurando que todas ellas sigan las mismas políticas de autenticación y gestión de usuarios. Esto promueve la consistencia y facilita el cumplimiento de normativas y estándares de seguridad.
    \item \textbf{Mejora de la Experiencia del Usuario:} Al ofrecer una autenticación rápida y segura, \textbf{CentralAuth} mejora la experiencia del usuario final, lo que puede aumentar la satisfacción y retención de usuarios en las aplicaciones que integren este servicio.
\end{itemize}

\subsection{Resumen de los Puntos Principales del Documento}

\begin{itemize}
    \item \textbf{Visión del Proyecto:} Descripción inspiradora del objetivo a largo plazo del proyecto y cómo contribuirá al éxito de la organización.
    \item \textbf{Introducción:} Contexto del proyecto, objetivos generales y específicos, y alcance del proyecto.
    \item \textbf{Descripción del Proyecto:} Detalles técnicos del proyecto, funcionalidades clave, tecnologías a utilizar y arquitectura del sistema.
    \item \textbf{Planificación del Proyecto:} Cronograma detallado, fases del proyecto y entregables de cada fase.
    \item \textbf{Recursos y Roles:} Equipo del proyecto y sus responsabilidades, y recursos necesarios.
    \item \textbf{Análisis de Riesgos:} Identificación de riesgos potenciales y plan de mitigación y contingencia.
    \item \textbf{Requisitos del Sistema:} Requisitos funcionales, no funcionales y de hardware y software.
    \item \textbf{Diseño del Sistema:} Diagramas de flujo, de entidad-relación, de clases y arquitectura de la base de datos.
    \item \textbf{Plan de Desarrollo:} Metodología de desarrollo, estrategias de implementación, plan de integración y pruebas.
    \item \textbf{Plan de Pruebas:} Tipos de pruebas a realizar, casos de prueba y plan de gestión de defectos.
    \item \textbf{Plan de Implementación:} Estrategias de despliegue, plan de capacitación para usuarios finales y plan de soporte post-implementación.
    \item \textbf{Plan de Mantenimiento:} Estrategias de mantenimiento y actualización, y plan de gestión de cambios.
    \item \textbf{Conclusiones:} Resumen de los puntos clave y próximos pasos.
    \item \textbf{Anexos:} Documentación adicional y referencias.
\end{itemize}

\newpage

\end{document}
