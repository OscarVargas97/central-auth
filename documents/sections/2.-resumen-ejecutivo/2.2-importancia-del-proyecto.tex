\subsection{Importancia del Proyecto}

El proyecto \textbf{CentralAuth} es crucial por varias razones:

\begin{itemize}
    \item \textbf{Seguridad Mejorada:} Al centralizar la autenticación y gestión de usuarios, se pueden implementar medidas de seguridad robustas de manera consistente en todas las aplicaciones y servicios que utilicen el microservicio. Esto reduce el riesgo de brechas de seguridad y garantiza una protección uniforme de los datos de los usuarios.
    \item \textbf{Eficiencia Operativa:} Centralizar la administración de usuarios permite un manejo más eficiente de las identidades, roles y permisos. Los administradores pueden gestionar todos los usuarios desde un único punto, reduciendo la redundancia y el esfuerzo manual asociado con la gestión de múltiples sistemas de autenticación.
    \item \textbf{Escalabilidad y Flexibilidad:} Al ser un microservicio independiente, \textbf{CentralAuth} puede escalarse horizontalmente para manejar un aumento en la carga de trabajo, lo que es esencial para aplicaciones que experimentan un crecimiento rápido en el número de usuarios.
    \item \textbf{Reutilización y Consistencia:} Este microservicio puede ser reutilizado por diferentes aplicaciones dentro de una organización, asegurando que todas ellas sigan las mismas políticas de autenticación y gestión de usuarios. Esto promueve la consistencia y facilita el cumplimiento de normativas y estándares de seguridad.
    \item \textbf{Mejora de la Experiencia del Usuario:} Al ofrecer una autenticación rápida y segura, \textbf{CentralAuth} mejora la experiencia del usuario final, lo que puede aumentar la satisfacción y retención de usuarios en las aplicaciones que integren este servicio.
\end{itemize}