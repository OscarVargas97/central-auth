\section{Introducción}

\subsection{Contexto del Proyecto}

En la actualidad, muchas organizaciones enfrentan desafíos relacionados con la gestión de usuarios y la autenticación debido a la existencia de múltiples sistemas independientes. Esta fragmentación genera problemas de seguridad, inconsistencias en la experiencia del usuario y un aumento en los costos operativos. Además, la integración de nuevas aplicaciones y servicios a menudo se complica por la falta de un sistema centralizado y eficiente para la gestión de identidades. "CentralAuth" surge como una solución a estos problemas, proporcionando un microservicio centralizado que simplifica y fortalece la autenticación y gestión de usuarios.

\subsection{Objetivos del Proyecto}

\textbf{Objetivo General:}

Desarrollar un microservicio centralizado para la autenticación y gestión de usuarios, que ofrezca seguridad, escalabilidad y eficiencia en la administración de identidades.

\textbf{Objetivos Específicos:}

\begin{itemize}
    \item Centralización de Datos: Unificar la gestión de usuarios y autenticación en un único sistema centralizado para facilitar el mantenimiento y la administración.
    \item Seguridad: Implementar medidas de seguridad avanzadas, incluyendo cifrado de contraseñas y autenticación de dos factores (2FA).
    \item Escalabilidad: Diseñar el sistema de manera que pueda escalar horizontalmente para manejar un gran número de usuarios y solicitudes simultáneas.
    \item Interoperabilidad: Garantizar que el microservicio pueda integrarse fácilmente con diversas aplicaciones y plataformas mediante API RESTful.
    \item Experiencia de Usuario: Asegurar una experiencia de usuario intuitiva y fluida tanto para los administradores como para los usuarios finales.
\end{itemize}

\subsection{Alcance del Proyecto}

El alcance del proyecto "CentralAuth" abarca las siguientes actividades:
\begin{itemize}
    \item Diseño del Sistema: Crear la arquitectura del sistema y definir los componentes clave.
    \item Desarrollo del Microservicio: Implementar las funcionalidades principales.
    \item Pruebas: Realizar pruebas unitarias, de integración y de aceptación para asegurar la calidad del microservicio.
    \item Documentación: Documentar la API y proporcionar guías de integración para desarrolladores.
    \item Despliegue: Implementar el microservicio en un entorno de producción y asegurar su disponibilidad.
\end{itemize}

\subsection{Alcance del Producto}

El alcance del producto "CentralAuth" incluye las siguientes funcionalidades principales:
\begin{itemize}
    \item Registro de Usuarios: Permitir a los usuarios crear una cuenta con su información básica.
    \item Inicio de Sesión: Autenticar a los usuarios mediante credenciales (correo electrónico y contraseña).
    \item Recuperación de Contraseñas: Permitir a los usuarios restablecer su contraseña mediante un correo electrónico de recuperación.
    \item Generación de Tokens JWT: Emitir tokens JWT para la autenticación segura de los usuarios en las aplicaciones conectadas.
    \item Verificación de Correo Electrónico: Enviar correos electrónicos de verificación para confirmar la dirección de correo electrónico de los usuarios.
\end{itemize}

\subsection{Estructura del Documento}

